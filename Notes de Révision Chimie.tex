\documentclass[french, a4paper, 12pt]{article}

%French setting up
\usepackage[utf8]{inputenc}
\usepackage[T1]{fontenc}
\usepackage{babel}

\setlength{\parindent}{0cm}
\usepackage{amsmath}
\usepackage{siunitx} %physics units
\usepackage{graphicx} % Images
\graphicspath{ {./images_chem/} }
\usepackage{fancyhdr}
\usepackage{float}

%Border
\usepackage[left=1in, right=1in, top=1in, bottom=1in]{geometry}
 
\pagestyle{fancy}
\fancyhf{}
\rhead{ImmortalPharaoh7}
\lhead{Notes de Révision Chimie NM}
\cfoot{\thepage}

\author{ImmortalPharaoh7}
\title{Notes de Révision Chimie NM}
\date{Mai 2020}

\usepackage{hyperref} %Last package
\urlstyle{same}
\begin{document}
\begin{titlepage}
\maketitle
\begin{abstract}
Les notes de révision pour la chimie niveau moyen du BI, pour le curriculum qui commence dès 2016. Attention, ces notes ne sont pas à être utilisées indépendamment; ils servent comme des astuces ou bien les définitions qui peuvent être oubliées.

Si vous avez des informations à ajouter ou bien des corrections, veuillez envoyer un email à pharaoh.immortal7@gmail.com ou messager ImmortalPharaoh7\#7811 sur Discord.
\end{abstract}
\end{titlepage}

\pagenumbering{roman}
\tableofcontents
\pagebreak

\pagenumbering{arabic}

\section{Relations Stœchiométriques}
\pagebreak

\section{Structure Atomique}
\pagebreak

\section{Périodicité}
\pagebreak

\section{Liaison et Structure Chimique}
\pagebreak

\section{Thermochimie}
\pagebreak

\section{Cinétique Chimique}
\pagebreak

\section{Équilibre}
\pagebreak

\section{Acides et Bases}
\pagebreak

\section{Processus Redox}
\pagebreak

\section{Chimie Organique}
\pagebreak

\section{Mesure et Traitement des Données}
\pagebreak

\end{document}