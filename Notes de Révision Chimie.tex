\documentclass[french, a4paper, 12pt]{article}

%French setting up
\usepackage[utf8]{inputenc}
\usepackage{lmodern}
\usepackage[T1]{fontenc}
\usepackage{babel}

\setlength{\parindent}{0cm}
\usepackage{amsmath}
\usepackage{siunitx} %physics units
\usepackage{graphicx} % Images
\graphicspath{ {./images_chem/} }
\usepackage{fancyhdr}
\usepackage{float}
\usepackage[version=4]{mhchem} %Chemical equations
\newcommand*\cec[1]{\cesplit{{\,\ }{\0}}{#1}} %For commas

%Border
\usepackage[left=1in, right=1in, top=1in, bottom=1in]{geometry}
 
\pagestyle{fancy}
\fancyhf{}
\rhead{ImmortalPharaoh7}
\lhead{Notes de Révision Chimie NM}
\cfoot{\thepage}

\author{ImmortalPharaoh7}
\title{Notes de Révision Chimie NM}
\date{Mai 2020}

\usepackage{hyperref} %Last package
\urlstyle{same}
\begin{document}
\begin{titlepage}
\maketitle
\begin{abstract}
Les notes de révision pour la chimie niveau moyen du BI, pour le curriculum qui commence dès 2016. Attention, ces notes ne sont pas à être utilisées indépendamment; ils servent comme des astuces ou bien les définitions qui peuvent être oubliées.

Si vous avez des informations à ajouter ou bien des corrections, veuillez envoyer un email à pharaoh.immortal7@gmail.com ou messager ImmortalPharaoh7\#7811 sur Discord.
\end{abstract}
\end{titlepage}

\pagenumbering{roman}
\tableofcontents
\pagebreak

\pagenumbering{arabic}

\section{Relations Stœchiométriques}
\pagebreak

\section{Structure Atomique}
\pagebreak

\section{Périodicité}
\pagebreak

\section{Liaison et Structure Chimique}
\pagebreak

\section{Thermochimie}
\textbf{Définitions:}
\begin{itemize}
\item Enthalpie: Énergie emmagasinée dans un matériel.
\item Endothermique: Une réaction qui absorbe de l'enthalpie $\Delta H > 0$.
\item Exothermique: Une réaction qui libère de l'enthalpie $\Delta H < 0$.
\item Conditions standards: Température de \SI{298}{K} et pression de \SI{100}{kPa}.
\item Calorimètre: Appareil qui permet de crée un système isolé au niveau de la température, mais il n'est jamais parfait.
\end{itemize}
\vspace{0.5em}
\textbf{Formules:}
\begin{itemize}
\item $Q=mc\Delta T$: $Q$ est l'énergie, $m$ la masse, $c$ est une constante et $\Delta T$ est la différence de température.
\item $\Delta H = -Q/n$: $\Delta H$ est l'enthalpie en \si{kJ.mol^{-1}}, $Q$ est l'énergie et $n$ est le nombre de mols.
\end{itemize}
\vspace{0.5em}
\textbf{Enthalpie moyenne des liaisons (section 11):}
\begin{align*}
\Delta H &= \text{liaisons détruites }-\text{ liaisons formées}\\
&=H_{initiale} - H_{finale}
\end{align*}
Attention: Ces valeurs sont des valeurs moyennes et \emph{tous} les composants doivent être en état gazeux.

\vspace{0.5em}
\textbf{Loi de Hess:}\\
Si une réaction chimique est la somme algébrique de plusieurs réactions, la chaleur de cette réaction est égale à la somme algébrique des chaleurs des réactions qui ont servi à établir cette somme. Il est possible d'inverser ou de multiplier les réactions pour enfin faire la somme algébrique.

Exemple: Trouver l'enthalpie dans la réaction suivante:
\[
\ce{2C2H6(g) + 7 O2(g) -> 4CO2(g) + 6H2O(g)}
\]
Avec les equations réactions suivantes:
\[
\left\{
	\begin{array}{ll}
		\ce{2C(s) + 3H2(g) -> C2H6(g)}\quad \Delta H=+84.7\\
		\ce{C(s) + O2(g) -> CO2(g)}\qquad \Delta H=+393.5\\
		\ce{H2(g) + 1/2O2 (g) -> H2O(g)} \quad \Delta H=+241.8
	\end{array}
\right.
\]
Donc il faut
\[
\text{Inverser:}\left\{
	\begin{array}{ll}
		\ce{C2H6(g) -> 2C(s) + 3H2(g)}\quad \Delta H=-84.7\\
		\ce{C(s) + O2(g) -> CO2(g)}\qquad \Delta H=+393.5\\
		\ce{H2(g) + 1/2O2 (g) -> H2O(g)} \quad \Delta H=+241.8
	\end{array}
\right.
\]
Ensuite
\[
\text{Multiplier:}\left\{
	\begin{array}{ll}
		2(\ce{C2H6(g) -> 2C(s) + 3H2(g)}\quad \Delta H=-84.7)\\
		4(\ce{C(s) + O2(g) -> CO2(g)}\qquad \Delta H=393.5)\\
		6(\ce{H2(g) + 1/2O2 (g) -> H2O(g)} \quad \Delta H=241.8)
	\end{array}
\right.
\]
Et enfin additionner les réactions avec leurs enthalpies pour donc avoir une enthalpie $\Delta H=\SI{2855.4}{kJ}$.

\vspace{0.5em}
\textbf{Chaleur de la formation standard (section 12):}
La variation d'enthalpie lors de la formation d'\emph{une} mole du composé à partir de ses éléments à l'état standard, ex:
\[
\ce{H2(g) + 1/2O2(g) -> H2O(l)}
\]

\vspace{0.5em}
\textbf{Chaleur de la combustion standard (section 13):}
La variation d'enthalpie lors de la combustion complète d'\emph{une} mole de la matière dans les conditions standards, ex:
\[
\ce{CO(g) + 1/2O2(g) -> CO2 (g)}
\]

\vspace{0.5em}
\textbf{Enthalpie de neutralisation:}
La variation d'enthalpie durant la formation d'une mole de H2O lors de la neutralisation de l'acide avec la base.
\[
\Delta H=\frac{-Q}{n_{limitant}}
\]
Attention: La combustion est incomplète et une partie de la chaleur s'échappe dans le milieu (calorimètre n'est jamais parfait).
\pagebreak

\section{Cinétique Chimique}
\textbf{Définitions:}
\begin{itemize}
\item Énergie d'activation: Énergie nécessaire pour que la réaction ait lieu.
\item Théorie des collisions: Pour la réaction ait lieu, il faut qu'il y a une collision avec une orientation géométrique appropriée et avec de l'énergie suffisante (énergie d'activation). 
\end{itemize}

\vspace{0.5em}
\textbf{Mesurer la vitesse de réaction:}
Faire une droite tangente au point qu'on veut mesurer sa vitesse, ensuite calculer le gradient de cette droite. Le résultat est la vitesse de réaction à ce point avec l'unité de l'ordonnée sur l'unité des abscisses.

\vspace{0.5em}
\textbf{Facteurs qui affectent la vitesse de réaction:}
\begin{itemize}
\item Surface exposée à la réaction: Plus la surface est grande (donc la matière est découpée en plus de parties), alors la vitesse de la réaction est grande.
\item Concentration: Plus la concentration est grande, plus la vitesse de réaction est grande.
\item Température: Plus la température est grande, plus la vitesse de réaction est grande.
\item Catalyseur: Il augmente la vitesse de réaction en diminuant l'énergie d'activation.
\end{itemize}

\begin{figure}[H]
\centering
\includegraphics[scale=0.9]{activation_energy}
\caption{Graphe de l'énergie d'activation}
\end{figure}

\vspace{0.5em}
\textbf{Distribution de Maxwell-Boltzmann:}
Seulement une fraction petite des particules on une énergie cinétique suffisante pour faire la réaction, le catalyseur augmente la fraction qui sont au dessus de l'énergie d'activation et la température change la courbe elle-même (voir les graphes).

\begin{figure}[H]
\centering
\includegraphics[scale=0.8]{maxwell-boltzmann}
\caption{Distribution normale qui doit être dessinée}
\end{figure}

\begin{figure}[H]
\centering
\includegraphics[scale=0.8]{maxwell-boltzmann_temperature}
\caption{Changement de la courbe à cause d'une augmentation de la température}
\end{figure}
\pagebreak

\section{Équilibre}
\textbf{Définitions:}
\begin{itemize}
\item Réaction réversible: Réaction dans laquelle les produits peuvent réagir pour produire les réactifs (la réaction se dirige dans les 2 sens).
\item Équilibre chimique: Quand la vitesse directe est égale à la vitesse inverse.
\end{itemize}

\vspace{0.5em}
\textbf{Facteurs qui affectent la position d'équilibre}
\begin{itemize}
\item Concentration: La concentration augmente la vitesse de réaction, alors une augmentation de la concentration des réactifs favorisera le sens direct.
\item Pression: Dans les gazes, une augmentation de pression favorisera le sens où il y a moins de mols.
\item Température: Dans une réaction endothermique, une augmentation de la température favorisera le sens direct. Une diminution favorisera le sens inverse. Vice versa pour une réaction exothermique.
\end{itemize}
Attention: Ajouter un catalyseur n'aura pas d'effet sur la position d'équilibre car il augmente la vitesse de réaction des 2 sens.

\vspace{0.5em}
\textbf{Quotient de la réaction et constant d'équilibre:}
Le quotient de réaction $Q_c$ est le rapport entre les produits et les réactifs mesuré à n'importe quelle point de la réaction. Le constant d'équilibre est le rapport entre les produits et les réactifs quand l'équation est à l'équilibre. Le \emph{seul} facteur qui peut changer le constant d'équilibre est un changement de température.\\
Attention: Les coefficients des molécules doivent être mis en puissance. Ex:
\[
\ce{2SO2(g) + O2(g) <=> 2SO3(g)} \implies K_c = \frac{[\ce{SO3}]^2}{[\ce{SO2}]^2[\ce{O2}]}
\]
\pagebreak

\section{Acides et Bases}
\textbf{Définitions:}
\begin{itemize}
\item Acide: Selon la théorie de Bronsted-Lowry, c'est une matière qui perd le proton \ce{H+}. Ex: \cec{HCl, HCO3, H2SO4}.
\item Base: Selon la théorie de Bronsted-Lowry, c'est une matière qui reçoit le proton \ce{H+}. Ex: \cec{NaOH, KOH, NH4OH}.
\item Amphiprotique: Une matière qui peut réagir comme un acide ou un base de Bronsted-Lowry selon le milieu.
\item Acide/Base conjuguée: La matière qui est produite après la réaction de la base/de l'acide. Un paire s'appelle une paire acide-base conjuguée. Ex: \ce{H2O}
\item Acide/Base fort(e): Acide/base qui est complètement ionisée dans l'eau. La réaction est irréversible, il sont des bons conducteurs d'électricité et leur paire conjuguée est faible. Ex: Acide: \ce{HCl, H2SO4, HNO3}, Base: \ce{NaOH, KOH}.
\item Acide/Base fort(e): Acide/base qui est partiellement ionisée dans l'eau. La réaction peut être réversible et ils sont des mauvais conducteurs d'électricité. Ex: Acide: \ce{H2CO3, H3PO4} et tous les acides organique, Base: \ce{NH4OH}.
\item pH: Puissance d'hydrogène = $-\log[\ce{H+}]$ et $pH + pOH = 14$. Plus le pH est petit, plus la solution est acide.
\end{itemize}

\vspace{0.5em}
\textbf{Réactions avec les acides:}

\begin{center}
1. Acide + Métal $\rightarrow$ Sel + Hydrogène. Ex:
\[
\ce{2HCl + Mg -> MgCl2 + H2}
\]

2. Acide + Oxyde Métallique $\rightarrow$ Sel + Eau. Ex:
\[
\ce{2HCl + MgO -> MgCl2 + H2O}
\]

3. Acide + Carbonate/Hydrogénocarbonate $\rightarrow$ Sel + Eau + Dioxyde de carbone. Ex:
\[
\ce{2HCl + Na2CO3 -> 2NaCl + H2O + CO2}
\]

4. Acide + Base $\rightarrow$ Sel + Eau. Ex:
\[
\ce{HCl + NaOH -> NaCl + H2O}
\]
\end{center}

\vspace{0.5em}
\textbf{Pluies acides:}
Ils sont les pluies qui sont polluées et donc endommagent les plantes et les bâtiments. Ces pluies sont formées d'acide nitrique ou d'acide sulfurique. Les pluies sont naturellement acide à cause du dioxyde de carbone, mais ceci a un pH = 5.6 car c'est un acide faible.
\[
\ce{CO2 + H2O -> H2CO3}
\]

Oxydation de l'azote:
\begin{align*}
\ce{2NO + O2 &-> 2NO2\\
2NO2 + H2O &-> HNO3 + HNO2\\
HNO2 + O2 &-> 2HNO3}
\end{align*}

Oxydation de souffre:
\begin{align*}
\ce{2SO2 + O2 &-> 2SO3\\
SO3 + H2O &-> H2SO4}
\end{align*}

Réaction avec le calcaire (les bâtiments):
\[
\ce{H2SO4 + CaCO3 -> CaSO4 + H2O + CO2}
\]

\vspace{0.5em}
\textbf{Méthodes pour diminuer les pluies acides:}\\
Pré-combustion: Broyer le charbon et la mettre dans l'eau et par flottement on diminue le souffre.\\
Post-combustion: Éliminer ces oxydes à l'aide des réaction chimique avec la chaux:
\[
\ce{CaO + SO2 -> CaSO3}
\]
pour éliminer les oxydes de souffre avant d'être libérés dans l'air.
\pagebreak

\section{Processus Redox}
\pagebreak

\section{Chimie Organique}
\pagebreak

\section{Mesure et Traitement des Données}
\pagebreak

\end{document}