\documentclass[french, a4paper, 12pt]{article}

%French setting up
\usepackage[utf8]{inputenc}
\usepackage[T1]{fontenc}
\usepackage{babel}

\setlength{\parindent}{0cm}
\usepackage{amsmath}
\usepackage{siunitx} %physics units
\usepackage{graphicx} % Images
\graphicspath{ {./images_chem/} }
\usepackage{fancyhdr}
\usepackage{float}
\usepackage[version=4]{mhchem} %Chemical equations

%Border
\usepackage[left=1in, right=1in, top=1in, bottom=1in]{geometry}
 
\pagestyle{fancy}
\fancyhf{}
\rhead{ImmortalPharaoh7}
\lhead{Notes de Révision Chimie NM}
\cfoot{\thepage}

\author{ImmortalPharaoh7}
\title{Notes de Révision Chimie NM}
\date{Mai 2020}

\usepackage{hyperref} %Last package
\urlstyle{same}
\begin{document}
\begin{titlepage}
\maketitle
\begin{abstract}
Les notes de révision pour la chimie niveau moyen du BI, pour le curriculum qui commence dès 2016. Attention, ces notes ne sont pas à être utilisées indépendamment; ils servent comme des astuces ou bien les définitions qui peuvent être oubliées.

Si vous avez des informations à ajouter ou bien des corrections, veuillez envoyer un email à pharaoh.immortal7@gmail.com ou messager ImmortalPharaoh7\#7811 sur Discord.
\end{abstract}
\end{titlepage}

\pagenumbering{roman}
\tableofcontents
\pagebreak

\pagenumbering{arabic}

\section{Relations Stœchiométriques}
\pagebreak

\section{Structure Atomique}
\pagebreak

\section{Périodicité}
\pagebreak

\section{Liaison et Structure Chimique}
\pagebreak

\section{Thermochimie}
\textbf{Définitions:}
\begin{itemize}
\item Enthalpie: Énergie emmagasinée dans un matériel.
\item Endothermique: Une réaction qui absorbe de l'enthalpie $\Delta H > 0$.
\item Exothermique: Une réaction qui libère de l'enthalpie $\Delta H < 0$.
\item Conditions standards: Température de \SI{298}{K} et pression de \SI{100}{kPa}.
\item Calorimètre: Appareil qui permet de crée un système isolé au niveau de la température, mais il n'est jamais parfait.
\end{itemize}
\vspace{0.5em}
\textbf{Formules:}
\begin{itemize}
\item $Q=mc\Delta T$: $Q$ est l'énergie, $m$ la masse, $c$ est une constante et $\Delta T$ est la différence de température.
\item $\Delta H = -Q/n$: $\Delta H$ est l'enthalpie en \si{kJ.mol^{-1}}, $Q$ est l'énergie et $n$ est le nombre de mols.
\end{itemize}
\vspace{0.5em}
\textbf{Enthalpie moyenne des liaisons (section 11):}
\begin{align*}
\Delta H &= \text{liaisons détruites }-\text{ liaisons formées}\\
&=H_{initiale} - H_{finale}
\end{align*}
Attention: Ces valeurs sont des valeurs moyennes et \emph{tous} les composants doivent être en état gazeux.

\vspace{0.5em}
\textbf{Loi de Hess:}\\
Si une réaction chimique est la somme algébrique de plusieurs réactions, la chaleur de cette réaction est égale à la somme algébrique des chaleurs des réactions qui ont servi à établir cette somme. Il est possible d'inverser ou de multiplier les réactions pour enfin faire la somme algébrique.

Exemple: Trouver l'enthalpie dans la réaction suivante:
\[
\ce{2C2H6(g) + 7 O2(g) -> 4CO2(g) + 6H2O(g)}
\]
Avec les equations réactions suivantes:
\[
\left\{
	\begin{array}{ll}
		\ce{2C(s) + 3H2(g) -> C2H6(g)}\quad \Delta H=+84.7\\
		\ce{C(s) + O2(g) -> CO2(g)}\qquad \Delta H=+393.5\\
		\ce{H2(g) + 1/2O2 (g) -> H2O(g)} \quad \Delta H=+241.8
	\end{array}
\right.
\]
Donc il faut
\[
\text{Inverser:}\left\{
	\begin{array}{ll}
		\ce{C2H6(g) -> 2C(s) + 3H2(g)}\quad \Delta H=-84.7\\
		\ce{C(s) + O2(g) -> CO2(g)}\qquad \Delta H=+393.5\\
		\ce{H2(g) + 1/2O2 (g) -> H2O(g)} \quad \Delta H=+241.8
	\end{array}
\right.
\]
Ensuite
\[
\text{Multiplier:}\left\{
	\begin{array}{ll}
		2(\ce{C2H6(g) -> 2C(s) + 3H2(g)}\quad \Delta H=-84.7)\\
		4(\ce{C(s) + O2(g) -> CO2(g)}\qquad \Delta H=393.5)\\
		6(\ce{H2(g) + 1/2O2 (g) -> H2O(g)} \quad \Delta H=241.8)
	\end{array}
\right.
\]
Et enfin additionner les réactions avec leurs enthalpies pour donc avoir une enthalpie $\Delta H=\SI{2855.4}{kJ}$.

\vspace{0.5em}
\textbf{Chaleur de la formation standard (section 12):}
La variation d'enthalpie lors de la formation d'\emph{une} mole du composé à partir de ses éléments à l'état standard, ex:
\[
\ce{H2(g) + 1/2O2(g) -> H2O(l)}
\]

\vspace{0.5em}
\textbf{Chaleur de la combustion standard (section 13):}
La variation d'enthalpie lors de la combustion complète d'\emph{une} mole de la matière dans les conditions standards, ex:
\[
\ce{CO(g) + 1/2O2(g) -> CO2 (g)}
\]

\vspace{0.5em}
\textbf{Enthalpie de neutralisation:}
La variation d'enthalpie durant la formation d'une mole de H2O lors de la neutralisation de l'acide avec la base.
\[
\Delta H=\frac{-Q}{n_{limitant}}
\]
Attention: La combustion est incomplète et une partie de la chaleur s'échappe dans le milieu (calorimètre n'est jamais parfait).
\pagebreak

\section{Cinétique Chimique}
\pagebreak

\section{Équilibre}
\pagebreak

\section{Acides et Bases}
\pagebreak

\section{Processus Redox}
\pagebreak

\section{Chimie Organique}
\pagebreak

\section{Mesure et Traitement des Données}
\pagebreak

\end{document}